\documentclass[a4paper,10pt,twocolumn]{article}
\usepackage[T1]{fontenc}
\usepackage[utf8]{inputenc}
\usepackage[italian]{babel}
\usepackage[top=2.5cm, bottom=2.5cm, left=2.5cm, right=2.5cm]{geometry}
\usepackage{graphicx}
\graphicspath{{/home/egg/uni/root/RootSimulation/build/analysis/}}
\usepackage{amsmath}
\usepackage{booktabs}
\usepackage{caption}
\captionsetup[table]{position=top}
\captionsetup[figure]{position=bottom}
\usepackage{dblfloatfix}
\usepackage[hidelinks]{hyperref}

\title{Relazione ROOT}
\author{Nicolò Montalti}
\date{\today}

\begin{document}
\maketitle
\section{Introduzione}
Il programma si divide in due parti: una di generazione e una di analisi. Durante la generazione vengono generati $10^5$ gruppi di 100 particelle secondo proporzioni predefinite. Per ogni generazione vengono riempiti degli istogrammi con
In seguito gli istogrammi vengono analizzati, disegnati su delle canvas e salvati.

\section{Struttura del codice}
Alla base del programma ci sono tre classi: ParticleType, ResonanceType e Particle. La classe ParticleType contiene la massa, la carica e il nome di ogni tipo di particella. Dispone inoltre dei rispettivi getters e di un metodo Print che permette di stampare a schermo ogni informazione. La classe ResonanceType eredita da ParticleType, secondo la relazione "is-a", e aggiunge la larghezza di risonanza con il rispettivo getter. Sovrascrive inoltre il metodo Print, in modo da stampare anche quest'ultimo attributo.

La classe Particle aggiunge le informazioni cinematiche, cioè le tre componenti della quantità di moto. Contiene un array statico di puntatori ParticleType, modificabile grazie al metodo statico AddParticleType. Ogni istanza di Particle contiene un indice intero che indica il tipo di particella. In questo modo le informazioni caratteristiche del tipo di particella, come massa, nome e carica, vengono salvate in memoria un'unica volta, e non per ogni singola istanza di Particle. Oltre ai canonici getters e setters dispone di un metodo Decay2Body che permette ai $K^*$ di decadere in un pione e in un kaone.

\section{Generazione}
Inizialmente vengono generati $10^5$ eventi di 100 particelle secondo le proporzioni definite in tab.~\ref{tab:proporzioni} attraverso un blocco if-else. La quantità di moto è assegnata estraendone il modulo da una distribuzione esponenziale di media 1 e gli angoli polare e azimutale da due distribuzioni uniformi. Le particelle $K^*$ vengono fatte decadere con pari probabilità in un $\pi^+$ e in un $K^-$ o in $\pi^-$ e in un $K^+$.

\begin{table*}
    \caption{Particelle generate dal programma con le rispettive proporzioni}
    \label{tab:proporzioni}
    \centering
    \begin{tabular}{cccc}
        \toprule
        Particelle         & Carica (e) & Simbolo & Percentuale \\
        \midrule
        Pioni              & +1         & $\pi^+$ & 40\%        \\
        Pioni              & -1         & $\pi^-$ & 40\%        \\
        Kaoni              & +1         & $K^+$   & 5\%         \\
        Kaoni              & -1         & $K^-$   & 5\%         \\
        Protoni            & +1         & $p^+$   & 4.5\%       \\
        Protoni            & -1         & $p^-$   & 4.5\%       \\
        Kaoni di risonanza & 0          & $K^*$   & 1\%         \\
        \bottomrule
    \end{tabular}
\end{table*}

\section{Analisi}

\begin{table*}
    \caption{Abbondanza delle particelle generate}
    \label{tab:abbondanza}
    \centering
    \begin{tabular}{cccc}
        \toprule
        Specie  & Occorrenze osservate $(10^5)$ & Occorrenze attese $(10^5)$ \\
        \midrule
        $\pi^+$ & 39.99 $\pm$ 0.02              & 40                         \\
        $\pi^-$ & 39.99 $\pm$ 0.02              & 40                         \\
        $K^+$   & 5.000 $\pm$ 0.007             & 5.0                        \\
        $K^-$   & 5.004 $\pm$ 0.007             & 5.0                        \\
        $p^+$   & 4.504 $\pm$ 0.007             & 4.5                        \\
        $p^-$   & 4.507 $\pm$ 0.007             & 4.5                        \\
        $K^*$   & 0.998 $\pm$ 0.003             & 1.0                        \\
        \bottomrule
    \end{tabular}
\end{table*}


\begin{table*}
    \caption{Fit delle distribuzioni}
    \label{tab:fit}
    \centering
    \begin{tabular}{cccccc}
        \toprule
        Variabile           & Distribuzione & Parametro del fit          & $\chi^2$ & DOF & $\chi^2$ / DOF \\
        \midrule
        Angolo azimutale    & pol0          & $(10000 \pm 3) \cdot 10^2$ & 27.86    & 9   & 3.10           \\
        Angolo polare       & pol0          & $(10000 \pm 3) \cdot 10^2$ & 15.20    & 9   & 1.70           \\
        Modulo dell'impulso & expo          & 1.0002 $\pm$ 0.0003        & 20.48    & 18  & 1.14           \\
        \bottomrule
    \end{tabular}
\end{table*}

\begin{table*}
    \caption{Analisi dei decadimenti delle $K^*$}
    \label{tab:fit}
    \centering
    \begin{tabular}{p{5cm}cccc}
        \toprule
        Distribuzione gaussiana                                                                          & Media                 & Sigma ($10^{-2}$) & Ampiezza ($10^4$) & $\chi^2$ / DOF \\
        \midrule
        Massa invariante ottenuta da differenza delle combinazioni $\pi K$ di carica discorde e concorde & 0.90101 $\pm$ 0.00017 & 4.536 $\pm$ 0.011 & 2.072 $\pm$ 0.008 & 78.7           \\
        \midrule
        Massa invariante ottenuta dalla differenza delle combinazioni di carica discorde e concorde      & 0.89110 $\pm$ 0.00016 & 4.227 $\pm$ 0.011 & 2.016 $\pm$ 0.009 & 77.2           \\
        \midrule
        Massa invariante ottenuta dalle vere $K^*$                                                       & 0.89167 $\pm$ 0.00016 & 5.042 $\pm$ 0.011 & 1.316 $\pm$ 0.005 & 0.855          \\
        \bottomrule
    \end{tabular}
\end{table*}

\begin{figure*}
    \includegraphics[width=0.95\linewidth]{Generation.pdf}
    \caption{Istogrammi delle particelle generate e attese (in alto a sx), del modulo dell'impulso con fit esponenziale (in alto a dx) e degli angoli azimutali e polari con fit pol0 (in basso)}
    \label{fig:Generation}
\end{figure*}

\begin{figure*}
    \includegraphics[width=0.95\linewidth]{InvMass.pdf}
    \caption{Istogrammi della massa invariante ottenuta rispettivamente dalla differenza delle combinazioni $k\pi$ di carica uguale ed opposta (in alto a sx), dalla differenza di combinazioni di particelle di carica uguale ed opposta (in alto a dx) e dalle coppie $K\pi$ generate nei decadimenti delle $K^*$ (in basso a sx)}
    \label{fig:InvMass}
\end{figure*}

\end{document}
